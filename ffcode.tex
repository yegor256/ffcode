% (The MIT License)
%
% Copyright (c) 2021 Yegor Bugayenko
%
% Permission is hereby granted, free of charge, to any person obtaining a copy
% of this software and associated documentation files (the 'Software'), to deal
% in the Software without restriction, including without limitation the rights
% to use, copy, modify, merge, publish, distribute, sublicense, and/or sell
% copies of the Software, and to permit persons to whom the Software is
% furnished to do so, subject to the following conditions:
%
% The above copyright notice and this permission notice shall be included in all
% copies or substantial portions of the Software.
%
% THE SOFTWARE IS PROVIDED 'AS IS', WITHOUT WARRANTY OF ANY KIND, EXPRESS OR
% IMPLIED, INCLUDING BUT NOT LIMITED TO THE WARRANTIES OF MERCHANTABILITY,
% FITNESS FOR A PARTICULAR PURPOSE AND NONINFRINGEMENT. IN NO EVENT SHALL THE
% AUTHORS OR COPYRIGHT HOLDERS BE LIABLE FOR ANY CLAIM, DAMAGES OR OTHER
% LIABILITY, WHETHER IN AN ACTION OF CONTRACT, TORT OR OTHERWISE, ARISING FROM,
% OUT OF OR IN CONNECTION WITH THE SOFTWARE OR THE USE OR OTHER DEALINGS IN THE
% SOFTWARE.

\documentclass[12pt]{article}
\usepackage[T1]{fontenc}
\usepackage[tt=false,type1=true]{libertine}
\usepackage{amsmath}
\usepackage{multicol}
\usepackage{ffcode}
\usepackage{xcolor}
\usepackage{microtype}
\title{\ff{ffcode}: \LaTeX{} Package \\ for Fixed-Font Code Blocks}
\author{Yegor Bugayenko}
\date{0.0.0 00.00.0000}
\begin{document}
\pagenumbering{gobble}
\raggedbottom
\setlength{\parindent}{0pt}
\setlength{\columnsep}{32pt}
\setlength{\parskip}{6pt}
\maketitle

This package helps you write source code in your articles
and make sure it looks nice. Install it from CTAN and then
use like this (pay attention to the \ff{\char`\\ff} command
and the \ff{ffcode} environment):

\begin{multicols}{2}
\setlength{\parskip}{0pt}
\scriptsize
\raggedcolumns
\begin{verbatim}
\documentclass{article}
\usepackage[T1]{fontenc}
\usepackage{ffcode}
\begin{document}
The function \ff{fibo()} is recursive:
\begin{ffcode}
int fibo(int n) {
  if (n < 2) {
    return n; |$\label{ln:ret}$|
  }
  return fibo(n - 1) + fibo(n - 2);
}
\end{ffcode}
The line~\ref{ln:ret} terminates it.
\end{document}
\end{verbatim}

\columnbreak

The function \ff{fibo()} is recursive:

\begin{ffcode}
int fibo(int n) {
  if (n < 2) {
    return n; |$\label{ln:ret}$|
  }
  return fibo(n - 1) + fibo(n - 2);
}
\end{ffcode}

The line no.~\ref{ln:ret} terminates it.
\end{multicols}

You have to run \ff{pdflatex} with the \ff{--shell-escape} flag
in order to let \ff{minted} (the package we use) to run Pygments
and format the code. If you don't want this to happen,
just use the \ff{nopygments} option.

A pair of vertical lines decorate a TeX command inside the snippet.
If you want to print a single vertical line, use this:
\ff{|\char`\\vert|}.

If you want to omit the light gray frames around \ff{\char`\\ff}
texts, use the package option \ff{noframes}.

To omit the line numbers, use the \ff{nonumbers} option of the package.

By default, the numbering is continuous: line numbers start at the
first snippet and increment until the end of the document. If you
want them to start from one at each snippet, use \ff{nocn}
(stands for ``no continuous numbering'')
option of the package.

You can highlight some lines in your \ff{ffcode} environment,
or can use any other additional configuration parameters from
the \ff{minted} package:

\begin{multicols}{2}
\setlength{\parskip}{0pt}
\scriptsize
\raggedcolumns
\begin{verbatim}
\begin{ffcode*}{highlightlines={1,4-5}}
while (true) {
  print("Hello!")
  print("Enter your name:")
  scan(x)
  print("You name is " + x)
}
\end{ffcode*}
\end{verbatim}

\columnbreak

\begin{ffcode*}{highlightlines={7,10-11}}
while (true) {
  print("Hello!")
  print("Enter your name:")
  scan(x)
  print("You name is " + x)
}
\end{ffcode*}
\end{multicols}

Using this second argument of the \ff{ffcode*} (with the trailing asterisk),
you can provide any other options from the \ff{minted} package to the
snippet.

By the way, the package correctly formats low-height texts, for example, just
a dot: \ff{.}

More details about this package you can find
in the \ff{yegor256/ffcode} GitHub repository.

\end{document}