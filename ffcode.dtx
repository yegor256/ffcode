% \iffalse meta-comment
% (The MIT License)
%
% Copyright (c) 2021-2022 Yegor Bugayenko
%
% Permission is hereby granted, free of charge, to any person obtaining a copy
% of this software and associated documentation files (the 'Software'), to deal
% in the Software without restriction, including without limitation the rights
% to use, copy, modify, merge, publish, distribute, sublicense, and/or sell
% copies of the Software, and to permit persons to whom the Software is
% furnished to do so, subject to the following conditions:
%
% The above copyright notice and this permission notice shall be included in all
% copies or substantial portions of the Software.
%
% THE SOFTWARE IS PROVIDED 'AS IS', WITHOUT WARRANTY OF ANY KIND, EXPRESS OR
% IMPLIED, INCLUDING BUT NOT LIMITED TO THE WARRANTIES OF MERCHANTABILITY,
% FITNESS FOR A PARTICULAR PURPOSE AND NONINFRINGEMENT. IN NO EVENT SHALL THE
% AUTHORS OR COPYRIGHT HOLDERS BE LIABLE FOR ANY CLAIM, DAMAGES OR OTHER
% LIABILITY, WHETHER IN AN ACTION OF CONTRACT, TORT OR OTHERWISE, ARISING FROM,
% OUT OF OR IN CONNECTION WITH THE SOFTWARE OR THE USE OR OTHER DEALINGS IN THE
% SOFTWARE.
% \fi

% \CheckSum{0}
%
% \CharacterTable
%  {Upper-case    \A\B\C\D\E\F\G\H\I\J\K\L\M\N\O\P\Q\R\S\T\U\V\W\X\Y\Z
%   Lower-case    \a\b\c\d\e\f\g\h\i\j\k\l\m\n\o\p\q\r\s\t\u\v\w\x\y\z
%   Digits        \0\1\2\3\4\5\6\7\8\9
%   Exclamation   \!     Double quote  \"     Hash (number) \#
%   Dollar        \$     Percent       \%     Ampersand     \&
%   Acute accent  \'     Left paren    \(     Right paren   \)
%   Asterisk      \*     Plus          \+     Comma         \,
%   Minus         \-     Point         \.     Solidus       \/
%   Colon         \:     Semicolon     \;     Less than     \<
%   Equals        \=     Greater than  \>     Question mark \?
%   Commercial at \@     Left bracket  \[     Backslash     \\
%   Right bracket \]     Circumflex    \^     Underscore    \_
%   Grave accent  \`     Left brace    \{     Vertical bar  \|
%   Right brace   \}     Tilde         \~}

% \GetFileInfo{ffcode.dtx}
% \DoNotIndex{\endgroup,\begingroup,\let,\else,\fi,\newcommand,\newenvironment}

% \iffalse
%<*driver>
\ProvidesFile{ffcode.dtx}
%</driver>
%<package>\NeedsTeXFormat{LaTeX2e}
%<package>\ProvidesPackage{ffcode}
%<*package>
[0000-00-00 0.0.0 Fixed Font Code]
%</package>
%<*driver>
\documentclass{ltxdoc}
\usepackage[T1]{fontenc}
\usepackage[tt=false,type1=true]{libertine}
\usepackage{microtype}
\AddToHook{env/verbatim/begin}{\microtypesetup{protrusion=false}}
\usepackage{href-ul}
\usepackage{ffcode}
\usepackage{amsmath}
\usepackage{multicol}
\usepackage{xcolor}
\usepackage[dtx,runs=2]{docshots}
\PageIndex
\EnableCrossrefs
\CodelineIndex
\RecordChanges
\begin{document}
	\DocInput{ffcode.dtx}
	\PrintChanges
	\PrintIndex
\end{document}
%</driver>
% \fi

% \title{|ffcode|: \LaTeX{} Package \\ for Fixed-Font Code Blocks\thanks{The sources are in GitHub at \href{https://github.com/yegor256/ffcode}{yegor256/ffcode}}}
% \author{Yegor Bugayenko \\ \texttt{yegor256@gmail.com}}
% \date{\filedate, \fileversion}
%
% \maketitle
%
% \section{Introduction}
%
% This package helps you write source code in your articles
% and make sure it looks nice. Install it from CTAN and then
% use like this (pay attention to |\ff| command
% and |ffcode| environment):
% \begin{docshot}
% \documentclass{article}
% \usepackage{ffcode}
% \pagestyle{empty}
% \begin{document}
% The function \ff{fibo()} is recursive:
% \begin{ffcode}
% int fibo(int n) {
%   if (n < 2) {
%     return n; |$\label{ln:ret}$|
%   }
%   return fibo(n-1)+fibo(n-2);
% }
% \end{ffcode}
% The line no.~\ref{ln:ret}
% terminates it.
% \end{document}
% \end{docshot}

% \section{Package Options}

% \DescribeMacro{nopygments}
% You have to run |pdflatex| with |--shell-escape| flag
% in order to let |minted| (the package we use) to run Pygments
% and format the code. If you don't want this to happen,
% just use |nopygments| option.

% \DescribeMacro{noframes}
% If you want to omit the light gray frames around |\ff|
% texts, use the package option |noframes|.

% \DescribeMacro{nobars}
% To omit the vertical gray bar at the left side of each snippet,
% use |nobars| option of the package.

% \DescribeMacro{nonumbers}
% To omit the line numbers, use |nonumbers| option of the package.

% \DescribeMacro{nocn}
% By default, the numbering is continuous: line numbers start at the
% first snippet and increment until the end of the document. If you
% want them to start from one at each snippet, use |nocn|
% (stands for ``no continuous numbering'')
% option of the package.

% \section{Typesetting}

% By the way, the package correctly formats low-height texts, for example, just
% a dot: \ff{.}

% A pair of vertical lines decorate a TeX command inside the snippet.
% If you want to print a single vertical line, use this:
% \ff{|\char`\\vert|}.

% The command |\ff| behaves differently in math mode --- it doesn't
% add gray frames:
% \begin{docshot}
% \documentclass{article}
% \usepackage{ffcode}
% \usepackage{mathtools}
% \pagestyle{empty}
% \begin{document}
% \begin{equation*}
% x = \int_\ff{home}^N f(x).
% \end{equation*}
% \end{document}
% \end{docshot}

% \section{Line Highlighting}

% You can highlight some lines in your |ffcode| environment,
% or can use any other additional configuration parameters from
% |minted| package:
% \begin{docshot}
% \documentclass{article}
% \usepackage[paperwidth=3in]{geometry}
% \usepackage{ffcode}
% \pagestyle{empty}
% \begin{document}
% \begin{ffcode*}
%   {highlightlines={1,4-5}}
% while (true) {
%   print("Hello!")
%   print("Enter your name:")
%   scan(x)
%   print("You name is " + x)
% }
% \end{ffcode*}
% \end{document}
% \end{docshot}

% Using this second argument of |ffcode*| (with the trailing asterisk),
% you can provide any other options from |minted| package to the
% snippet.

% \StopEventually{}

% \section{Implementation}
% \changes{v0.1.0}{2021/06/10}{Initial version}

% First, we parse package options with the help of
% \href{https://ctan.org/pkg/pgfopts}{pgfopts} package:
% \changes{v0.2.0}{2021/06/13}{Package options \texttt{nonumbers} and \texttt{noframes} added.}
% \changes{v0.3.0}{2021/09/07}{Package option \texttt{nocn} added.}
% \changes{v0.4.0}{2022/01/09}{Package option \texttt{nobars} added.}
% \changes{v0.6.0}{2021/11/14}{We use \texttt{pgfopts} instead of \texttt{xkeyval}.}
%    \begin{macrocode}
\RequirePackage{pgfopts}
\pgfkeys{
  /ff/.cd,
  nopygments/.store in=\ff@nopygments,
  noframes/.store in=\ff@noframes,
  nonumbers/.store in=\ff@nonumbers,
  nobars/.store in=\ff@nobars,
  nocn/.store in=\ff@nocn,
}
\ProcessPgfOptions{/ff}
%    \end{macrocode}

% Then, we disable pygments for \href{https://ctan.org/pkg/pgfopts}{minted}, if necessary:
%    \begin{macrocode}
\makeatletter\ifdefined\ff@nopygments
  \PassOptionsToPackage{draft=true}{minted}
\fi\makeatother
%    \end{macrocode}

% Then, we configure \href{https://ctan.org/pkg/pgfopts}{minted} package:
%    \begin{macrocode}
\RequirePackage{minted}
\setminted{breaklines}
\setminted{escapeinside=||,mathescape}
\setminted{highlightcolor=gray!25}
\usemintedstyle{bw}
%    \end{macrocode}

% \begin{macro}{ffcode}
% Then, we define |ffcode| environment:
%    \begin{macrocode}
\makeatletter\ifdefined\ff@nonumbers
  \ifdefined\ff@nobars
    \newminted[ffcode]{text}{}
  \else
    \newminted[ffcode]{text}{framesep=6pt,
      framerule=1pt,rulecolor=gray,frame=leftline}
  \fi
\else
  \renewcommand{\theFancyVerbLine}{\textcolor{gray}%
    {\tiny\oldstylenums{\ttfamily\arabic{FancyVerbLine}}}}
  \ifdefined\ff@nocn
    \ifdefined\ff@nobars
      \newminted[ffcode]{text}{
        linenos,numbersep=2pt
      }
    \else
      \newminted[ffcode]{text}{
        framesep=6pt,framerule=1pt,rulecolor=gray,
        frame=leftline,linenos,numbersep=2pt
      }
    \fi
  \else
    \ifdefined\ff@nobars
      \newminted[ffcode]{text}{
        firstnumber=last,linenos,numbersep=2pt
      }
    \else
      \newminted[ffcode]{text}{
        framesep=6pt,framerule=1pt,rulecolor=gray,
        frame=leftline,firstnumber=last,linenos,numbersep=2pt
      }
    \fi
  \fi
\fi\makeatother
%    \end{macrocode}
% \end{macro}

% \begin{macro}{\ff@print}
% Then, we define supplementary commands |\ff@print| and |\ff@printx|:
% \changes{v0.5.1}{2022/10/30}{Now, the command \texttt{ff} ignores italic and bold and always prints \texttt{\char`\\texttt} as it should be.}
%    \begin{macrocode}
\makeatletter
\newcommand\ff@print[1]{\textnormal{\texttt{\detokenize{#1}}}}
\makeatother
%    \end{macrocode}
% \end{macro}

% \begin{macro}{\ff@rule}
% Then, we define supplementary command |\ff@rule|:
%    \begin{macrocode}
\makeatletter\newcommand\ff@rule
  {\vrule height 6pt depth 1pt width 0pt}
\makeatother
%    \end{macrocode}
% \end{macro}

% \begin{macro}{\ff}
% Then, we define |\ff| command:
%    \begin{macrocode}
\makeatletter\ifdefined\ff@noframes
  \newcommand\ff[1]{\ff@rule\ff@print{#1}}
\else
  \RequirePackage{tcolorbox}
  \newtcbox\ff@box{nobeforeafter,colframe=gray!80!white,
    colback=gray!5!white,boxrule=0.1pt,arc=1pt,
    boxsep=1.2pt,left=0.5pt,right=0.5pt,top=0.2pt,bottom=0.2pt,
    tcbox raise base}
  \newcommand\ff[1]{\relax\ifmmode%
    {\ff@rule\ff@print{#1}}%
  \else%
    {\ff@box{\ff@rule\ff@print{#1}}}%
  \fi}
\fi\makeatother
%    \end{macrocode}
% \end{macro}

% \begin{macro}{\ff}
% Finally, we let vertical bars work similar to |\ff|:
%\iffalse%
%<*verb>%
%\fi%
%    \begin{macrocode}
\catcode`|=\active
\def|#1|{\ff{#1}}
%\iffalse
%</verb>
%\fi
%    \end{macrocode}
% \end{macro}

% \Finale

%\clearpage
%
%\PrintChanges
%\clearpage
%\PrintIndex
